% explicitly call these packages to avoid this issue:
% https://stackoverflow.com/questions/46080853/why-does-rendering-a-pdf-from-rmarkdown-require-closing-rstudio-between-renders?
% utm_medium=organic&utm_source=google_rich_qa&utm_campaign=google_rich_qa
\usepackage{booktabs}
\usepackage{longtable}
\usepackage{array}
\usepackage{multirow}
\usepackage{color}          % enables font colors
\usepackage{wrapfig}
\usepackage{float}
\usepackage{colortbl}
\usepackage{pdflscape}
\usepackage{tabu}
\usepackage{threeparttable}
\usepackage[normalem]{ulem}
\usepackage[export]{adjustbox}  % enables two column images at the top
\usepackage{makecell}           % for library(kableExtra)
%\usepackage{tabularx}           % for library(kableExtra)
\usepackage{geometry}

% set font encoding for PDFLaTeX or XeLaTeX
\usepackage[utf8]{inputenc}   % sets document encoding to utf8
\usepackage[T1]{fontenc}
\usepackage{tikz}
\usetikzlibrary{calc}
\usepackage{memhfixc}
\usepackage{amssymb}          % for itemized list styles

% document setup
\definecolor{clablue}{HTML}{335eac}    % defines the ITA blue color
\usepackage{enumitem}                   % bullet alignment
\setlist[2]{nosep}                      % sets the itemsep and parsep for all level two lists to 0
\setenumerate{nosep}                    % sets no itemsep for enumerate lists only
% \pagenumbering{gobble}                  % disable page numbering
\usepackage[hang,flushmargin]{footmisc} % don't indent footnotes
\usepackage[document]{ragged2e}         % ragged right
\usepackage[none]{hyphenat}             % disable work breaks
\renewcommand\thefootnote{\textcolor{clablue}{\arabic{footnote}}} % change the color of the footnote indicator
\hypersetup{
  colorlinks,
  linkcolor=clablue,
  urlcolor=clablue,
}
\usepackage{parskip}
\setlength{\parskip}{0.1in}               % set paragraph spacing


\usepackage{geometry}
 \geometry{
 letterpaper,
 left=1in,
 top=1in,
 bottom=0.7in
 }

% urban institute logo
\newcommand{\clalogo}[0]{
  \begin{figure}
    \hspace{-0.12in}
    \includegraphics[width=1.5in]{design/cla_logo.png}
    \includegraphics[width=1.5in]{design/ita_logo.png}
    \vspace{0.25in}  
    \label{fig:boat1}
  \end{figure}
}

% contact information
\newcommand{\contactinfo}[0]{
  \thispagestyle{empty}
  \begin{tikzpicture}[remember picture, overlay]
    \draw 
      node[anchor=south west, minimum height=0.3in, minimum width=7.5in, text width=5in] at ($(current page.south west) + (0.45in,0.13in)$) 
      {\begin{center}
      \color{gray}City of Los Angeles - Information Technology Agency \space \normalsize\break
      \color{clablue}$\blacksquare$ \space
      \color{gray}200 N Main St \space
      \color{clablue}$\blacksquare$ \space
      \color{gray}Los Angeles, CA \space
      \color{clablue}$\blacksquare$ \space
      \color{gray}213.978.0874 \space
      \color{clablue}$\blacksquare$ \space
      \textcolor{clablue}{www.ita.lacity.org}
      \end{center}
      };
  \end{tikzpicture}
}

% titles (14pt font)
\newcommand{\clatitle}[1]{
  \begin{center}
      \textbf{\LARGE{#1}}
  \end{center}
}
  
% subtitles (12pt font)
\newcommand{\clasubtitle}[1]{
  \begin{center}
    \Large{\textcolor{clablue}{#1}}
    \vspace{0.25in}    
  \end{center}
}

% authors (11pt font)
\newcommand{\claauthors}[1]{
  \begin{center}
    \vspace{0.25in} 
  \end{center}
}

% heading 1
\newcommand{\claheadingone}[1]{
  \textbf{\LARGE{\\} \large{#1}}
}

% heading 2
\newcommand{\claheadingtwo}[1]{
  \textbf{\textcolor{clablue}{#1}}
}

% figure label
\newcommand{\clafigurenumber}[1]{
  \textcolor{clablue}{\tiny{\\} \normalsize{FIGURE #1 \\}}
}

% figure title
\newcommand{\clafiguretitle}[1]{
  \textbf{\normalsize{#1} \\}
}

% figure source
\newcommand{\clasource}[1]{
  \footnotesize{\textbf{Source: }#1} \normalsize\break
}

% figure note
\newcommand{\clanote}[1]{
  \footnotesize{\textbf{Note: }#1} \normalsize\break
}

% table label
\newcommand{\clatablenumber}[1]{
  \textcolor{clablue}{\tiny{\\} \normalsize{TABLE #1 \\}}
}

% table title
\newcommand{\clatabletitle}[1]{
  \textbf{\normalsize{#1} \\}
}

% table sub title
\newcommand{\clatablesubtitle}[1]{
  \textit{\normalsize{#1} \\}
}


% bullet points
\renewcommand{\labelitemi}{\color{clablue}\tiny$\blacksquare$}   % blue square
\renewcommand{\labelitemii}{\color{clablue}\tiny$\blacksquare$}  % blue square
\renewcommand{\labelitemiii}{\color{clablue}\tiny$\blacksquare$} % blue square
\renewcommand{\labelitemiv}{\color{clablue}\tiny$\blacksquare$}  % blue square

\newenvironment{clabullets}
{\begin{itemize}[leftmargin=*,labelindent=0.25in,labelsep=0.1875in]
\setlength{\itemsep}{0pt}
\setlength{\parskip}{0pt}
\setlength{\parsep}{0pt}}
{\end{itemize}} 

% numbered list
\newenvironment{claenumerate}
{\begin{enumerate}[leftmargin=*,labelindent=0.25in,labelsep=0.15625in]
\setlength{\itemsep}{0pt}
\setlength{\parskip}{0pt}
\setlength{\parsep}{0pt}}
{\end{enumerate}} 

% eliminates top border line for the table
\def\toprule{}

% sets midline at 1pt
\setlength{\lightrulewidth}{1pt}
